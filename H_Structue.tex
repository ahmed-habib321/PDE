%----------------------------------------------------------------------------------------
%	REQUIRED PACKAGES and commands
%----------------------------------------------------------------------------------------
\usepackage{titlesec} % Allows customization of titles
\usepackage{graphicx}        % Required for including pictures
\graphicspath{{Pictures/}}  % Specifies the directory where pictures are stored
\usepackage{tikz} % Required for drawing custom shapes
\usepackage{lipsum} 
\usepackage{enumitem} % Customize lists
\setlist{nolistsep} % Reduce spacing between bullet points and numbered lists
\usepackage{amsmath}
\usepackage{graphicx}
\usepackage{blindtext}
\usepackage{titlesec}
\usepackage[skip=9pt plus1pt, indent=0pt]{parskip}
\usepackage{wrapfig}
\usepackage{caption2}
\usepackage{flexisym}
\usepackage{comment}
\usepackage[export]{adjustbox}
\usepackage{tcolorbox}
\usepackage{mathrsfs}
\usepackage{siunitx}
\usepackage{tikz}
\usepackage{amssymb}
\usepackage{eso-pic}
\usepackage{mwe}
\hyphenpenalty=10000
\usepackage{xcolor}
\definecolor{Solution}{RGB}{0, 135, 189} 
\allowdisplaybreaks
\usetikzlibrary{positioning,calc}
\usetikzlibrary{patterns}
\newcommand*\circled[1]{\tikz[baseline=(char.base)]{
            \node[shape=circle,draw,inner sep=2pt] (char) {#1};}}
\usepackage{booktabs} % Required for nicer horizontal rules in tables
\usepackage{eso-pic} % Required for specifying an image background in the title page

%----------------------------------------------------------------------------------------
%	PDF theme color
%----------------------------------------------------------------------------------------

\definecolor{theme}{RGB}{35, 114, 39} % Define the color used for highlighting throughout the pdf

%----------------------------------------------------------------------------------------
%	MAIN TABLE OF CONTENTS
%----------------------------------------------------------------------------------------

\usepackage{titletoc} % Required for manipulating the table of contents

\contentsmargin{0cm} % Removes the default margin

% section text styling
\titlecontents{section}[1.25cm] % Indentation
{\addvspace{5pt}\large\bfseries} % Spacing and font options for sections
{\color{theme!70}\contentslabel[\Large\thecontentslabel]{1.25cm}\color{theme}} % section number
{}  
{\color{theme!70}\normalsize\bfseries\;\titlerule*[.5pc]{.}\;\thecontentspage} % Page number

% Subsection text styling
\titlecontents{subsection}[1.75cm] % Indentation
{\addvspace{1pt}\small} % Spacing and font options for subsections
{\contentslabel[\thecontentslabel]{1.25cm}} % Subsection number
{}
{\;\titlerule*[.5pc]{.}\;\thecontentspage} % Page number
[] 

% Exclude subsubsection from table of contents
\titlecontents{subsubsection}[2cm] % Indentation
{\addvspace{1pt}\small} % Spacing and font options for subsections
{\contentslabel[\thecontentslabel]{1.25cm}} % Subsection number
{}
{\;\titlerule*[.5pc]{.}\;\thecontentspage} % Page number
[]  



\usepackage{amsmath,amsfonts,amssymb,amsthm} % For math equations, theorems, symbols, etc
\newtheorem{notation}{Notation}[section]

%%%%%%%%%%%%%%%%%%%%%%%%%%%%%%%%%%%%%%%%%%%%%%%%%%%%%%%%%%%%%%%%%%%%%%%%%%%
%%%%%%%%%%%%%%%%%%%% dedicated to boxed/framed environements %%%%%%%%%%%%%%
%%%%%%%%%%%%%%%%%%%%%%%%%%%%%%%%%%%%%%%%%%%%%%%%%%%%%%%%%%%%%%%%%%%%%%%%%%%
\newtheoremstyle{themenumbox}% % Theorem style name
{0pt}% Space above
{0pt}% Space below
{\normalfont}% % Body font
{}% Indent amount
{\small\bf\color{theme}}% % Theorem head font
{\;}% Punctuation after theorem head
{0.25em}% Space after theorem head
{\small\color{theme}\thmname{#1}\nobreakspace\thmnumber{#2}% Theorem text (e.g. Theorem 2.1)
\thmnote{\nobreakspace\textit\bfseries\color{black}---\nobreakspace#3.}} % Optional theorem note
\renewcommand{\qedsymbol}{$\blacksquare$}% Optional qed square

\newtheoremstyle{blacknumex}% Theorem style name
{5pt}% Space above
{5pt}% Space below
{\normalfont}% Body font
{} % Indent amount
{\small\bf}% Theorem head font
{\;}% Punctuation after theorem head
{0.25em}% Space after theorem head
{\small{\tiny\ensuremath{\blacksquare}}\nobreakspace\thmname{#1}\nobreakspace\thmnumber{#2}% Theorem text (e.g. Theorem 2.1)
\thmnote{\nobreakspace\textit\bfseries---\nobreakspace#3.}}% Optional theorem note

\newtheoremstyle{blacknumbox} % Theorem style name
{0pt}% Space above
{0pt}% Space below
{\normalfont}% Body font
{}% Indent amount
{\small\bf}% Theorem head font
{\;}% Punctuation after theorem head
{0.25em}% Space after theorem head
{\small\thmname{#1}\nobreakspace\thmnumber{#2}% Theorem text (e.g. Theorem 2.1)
\thmnote{\nobreakspace\textit\bfseries---\nobreakspace#3.}}% Optional theorem note

%%%%%%%%%%%%%%%%%%%%%%%%%%%%%%%%%%%%%%%%%%%%%%%%%%%%%%%%%%%%%%%%%%%%%%%%%%%
%%%%%%%%%%%%% dedicated to non-boxed/non-framed environements %%%%%%%%%%%%%
%%%%%%%%%%%%%%%%%%%%%%%%%%%%%%%%%%%%%%%%%%%%%%%%%%%%%%%%%%%%%%%%%%%%%%%%%%%
\newtheoremstyle{themenum}% % Theorem style name
{5pt}% Space above
{5pt}% Space below
{\normalfont}% % Body font
{}% Indent amount
{\small\bf\color{theme}}% % Theorem head font
{\;}% Punctuation after theorem head
{0.25em}% Space after theorem head
{\small\color{theme}\thmname{#1}\nobreakspace\thmnumber{#2}% Theorem text (e.g. Theorem 2.1)
\thmnote{\nobreakspace\textit\bfseries\color{black}---\nobreakspace#3.}} % Optional theorem note
\renewcommand{\qedsymbol}{$\blacksquare$}% Optional qed square

\newtheoremstyle{enbox}% % Theorem style name
{0pt}% Space above
{0pt}% Space below
{}
{}
{}
{}
{0.25em}
{\small
\thmnote{\nobreakspace\textit\bfseries\color{black}---\nobreakspace#3.}} % Optional theorem note
\renewcommand{\qedsymbol}{$\blacksquare$}% Optional qed square


\makeatother

\newcounter{dummy}
\numberwithin{dummy}{section}
\theoremstyle{themenumbox}

\newtheorem{theoremeT}[dummy]{Theorem}

\newtheorem{lemma}[dummy]{Lemma}
\newtheorem{observation}[dummy]{Observation}
\newtheorem{proposition}[dummy]{Proposition}
% \newtheorem{definition}[dummy]{Definition}
\newtheorem{claim}[dummy]{Claim}
\newtheorem{fact}[dummy]{Fact}
\newtheorem{assumption}[dummy]{Assumption}

\newtheorem{problem}{Problem}[section]
% \newtheorem{exercise}{Exercise}[section]
\theoremstyle{blacknumex}
\newtheorem{exampleT}{Example}[subsection]
\theoremstyle{blacknumbox}
\newtheorem{vocabulary}{Vocabulary}[section]
\newtheorem{definitionT}{Definition}[section]
\newtheorem{corollaryT}[dummy]{Corollary}
\theoremstyle{themenum}



%----------------------------------------------------------------------------------------
%	DEFINITION OF COLORED BOXES
%----------------------------------------------------------------------------------------

\RequirePackage[framemethod=default]{mdframed} % Required for creating the theorem, definition, exercise and corollary boxes

% Theorem box
\newmdenv[
    skipabove=7pt,
    skipbelow=7pt,
    backgroundcolor=black!5,
    linecolor=theme,
    innerleftmargin=5pt,
    innerrightmargin=5pt,
    innertopmargin=5pt,
    leftmargin=0cm,
    rightmargin=0cm,
    innerbottommargin=5pt]{tBox}

% Exercise box	  
\newmdenv[skipabove=7pt,
skipbelow=7pt,
rightline=false,
leftline=true,
topline=false,
bottomline=false,
backgroundcolor=theme!10,
linecolor=theme,
innerleftmargin=5pt,
innerrightmargin=5pt,
innertopmargin=5pt,
innerbottommargin=5pt,
leftmargin=0cm,
rightmargin=0cm,
linewidth=4pt]{eBox}	

% Definition box
\newmdenv[
skipabove=7pt,
skipbelow=7pt,
rightline=false,
leftline=true,
topline=false,
bottomline=false,
linecolor=theme,
innerleftmargin=5pt,
innerrightmargin=5pt,
innertopmargin=0pt,
leftmargin=0cm,
rightmargin=0cm,
linewidth=4pt,
innerbottommargin=0pt]{dBox}	

% Corollary box
\newmdenv[skipabove=7pt,
skipbelow=7pt,
rightline=false,
leftline=true,
topline=false,
bottomline=false,
linecolor=gray,
backgroundcolor=black!5,
innerleftmargin=5pt,
innerrightmargin=5pt,
innertopmargin=5pt,
leftmargin=0cm,
rightmargin=0cm,
linewidth=4pt,
innerbottommargin=5pt]{cBox}

% Creates an environment for each type of theorem and assigns it a theorem text style from the "Theorem Styles" section above and a colored box from above
\newenvironment{theorem}{\begin{tBox}\begin{theoremeT}}{\end{theoremeT}\end{tBox}}
\newenvironment{exercise}{\begin{eBox}\begin{exerciseT}}{\hfill{\color{theme}\tiny\ensuremath{\blacksquare}}\end{exerciseT}\end{eBox}}				  
\newenvironment{definition}{\begin{dBox}\begin{definitionT}}{\end{definitionT}\end{dBox}}	
\newenvironment{example}{\begin{exampleT}}{\hfill{\tiny\ensuremath{\blacksquare}}\end{exampleT}}		
\newenvironment{corollary}{\begin{cBox}\begin{corollaryT}}{\end{corollaryT}\end{cBox}}	
\newenvironment{enrichment*}[1]{\begin{tcolorbox}[colback=theme!5!white,colframe=theme!50!black,title=#1]}{\end{tcolorbox}}



\newenvironment{enrichment}[5]{
    \def\Title{#1}    
    \def\imagepath{#2}
    \def\imagewidth{#3}
    \def\textspace{#4}
    \def\imagespace{#5}
    \begin{tcolorbox}[colback=theme!5!white,colframe=theme!50!black,title=\Title]
        \begin{minipage}[t]{\textspace\textwidth}
        }{
        \end{minipage}
        \hfill
        \begin{minipage}[t]{\imagespace\textwidth}
            \includegraphics[width=\imagewidth cm,valign=t]{\imagepath}
        \end{minipage}
    \end{tcolorbox}
}


%----------------------------------------------------------------------------------------
%	SECTION NUMBERING IN THE MARGIN
%----------------------------------------------------------------------------------------

\makeatletter
\renewcommand{\@seccntformat}[1]{\llap{\textcolor{theme}{\csname the#1\endcsname}\hspace{1em}}}                    
\renewcommand{\section}{\@startsection{section}{1}{\z@}
{-4ex \@plus -1ex \@minus -.4ex}
{1ex \@plus.2ex }
{\normalfont\large\bfseries}}
\renewcommand{\subsection}{\@startsection {subsection}{2}{\z@}
{-3ex \@plus -0.1ex \@minus -.4ex}
{0.5ex \@plus.2ex }
{\normalfont\bfseries}}
\renewcommand{\subsubsection}{\@startsection {subsubsection}{3}{\z@}
{-2ex \@plus -0.1ex \@minus -.2ex}
{.2ex \@plus.2ex }
{\normalfont\small\bfseries}}                        
\renewcommand\paragraph{\@startsection{paragraph}{4}{\z@}
{-2ex \@plus-.2ex \@minus .2ex}
{.1ex}
{\normalfont\small\bfseries}}




